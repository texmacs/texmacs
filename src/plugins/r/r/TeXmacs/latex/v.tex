\Header{view}{Insert current graphic into TeXmacs}
\alias{linev}{view}
\alias{plotv}{view}
\alias{pointsv}{view}
\alias{v}{view}
\keyword{TeXmacs}{view}
\keyword{graphics}{view}
\keyword{view}{view}
\keyword{device}{view}
\keyword{plot}{view}
\begin{Description}\relax
Once a graph is plotted and ready to go into the TeXmacs buffer,
you insert it into the buffer using the v() function. The plotv,
linev, and pointsv are just shortcuts for plot and then v.\end{Description}
\begin{Usage}
\begin{verbatim}
v(width = 4, height = 4, ...)
plotv(...)
pointsv(...)
linev(...)
\end{verbatim}
\end{Usage}
\begin{Arguments}
\begin{ldescription}
\item[\code{width}] Width of graph in inches. 
\item[\code{height}] Height of graph in inches. 
\item[\code{...}] Additional arguments for postscript. 
\end{ldescription}
\end{Arguments}
\begin{Author}\relax
Michael Lachmann Tamarlin\end{Author}
\begin{SeeAlso}\relax
See Also \code{\Link{TeXmacs}},\end{SeeAlso}
\begin{Examples}
\begin{ExampleCode}
x<-(0:600)/100;
plot(x,sin(x),type="l");
lines(x,cos(x),col=2);
legend(0,-0.5,c("sin(x)","cos(x)"),lty=1,col=c(1:2));
v();
plotv(x,sin(60/x),type="l");
\end{ExampleCode}
\end{Examples}

